%main <main.tex>
%next <purpose.tex>

%\usepackage{float} 
\usepackage[dvipdfmx]{graphicx}
%%数学関連
\usepackage{amsmath,amssymb,bm}

%\usepackage{subfigure}
%\usepackage{pdfpages}

\newcommand{\argmax}{\mathop{\rm arg~max}\limits}
\newcommand{\argmin}{\mathop{\rm arg~min}\limits}

%\renewcommand{\figurename}{Figure}

\renewcommand{\labelenumi}{(\theenumi)}

\usepackage[T1]{fontenc}
\usepackage{textcomp}
\usepackage{lmodern}

%%任意の場所で図や表のキャプションを定義
%%\figcaption もしくは \tabcaption で任意の場所でキャプションを定義できる
\makeatletter
\def\tbcaption{\def\@captype{table}\caption}
\def\figcaption{\def\@captype{figure}\caption}
\makeatother

\renewenvironment{thebibliography}[1]
{\section*{\refname\@mkboth{\refname}{\refname}}%
  \list{\@biblabel{\@arabic\c@enumiv}}%
       {\settowidth\labelwidth{\@biblabel{#1}}%
        \leftmargin\labelwidth
        \advance\leftmargin\labelsep
 \setlength\itemsep{-0.5zh}%←ここの数値を調整(行間のつまり具合)
 \setlength\baselineskip{11pt}%←ここの数値を調整(追加)(文字の大きさ)
        \@openbib@code
        \usecounter{enumiv}%
        \let\p@enumiv\@empty
        \renewcommand\theenumiv{\@arabic\c@enumiv}}%
  \sloppy
  \clubpenalty4000
  \@clubpenalty\clubpenalty
  \widowpenalty4000%
  \sfcode`\.\@m}
 {\def\@noitemerr
   {\@latex@warning{Empty `thebibliography' environment}}%
  \endlist}


%%%ソースコードの貼り付け
\usepackage{listings,jlisting}
%\renewcommand{\lstlistingname}{ソースコード目次}
%\renewcommand{\lstlistlistingname}{ソースコード}
%%Usage: \lstinputlisting[caption=キャプション,label=ラベル] {ファイルパス}
\lstset{%
 basicstyle={\small},%
 identifierstyle={\small},%
 commentstyle={\small\itshape},%
 keywordstyle={\small\bfseries},%
 ndkeywordstyle={\small},%
 stringstyle={\small\ttfamily},
 breaklines=true,
 columns=[l]{fullflexible},%
 numbers=left,%
 xrightmargin=0zw,%
 xleftmargin=3zw,%
 numberstyle={\scriptsize},%
 stepnumber=1,
 numbersep=1zw,%
 lineskip=-0.5ex,%
 showspaces=false,%
 captionpos=b,%
 tabsize=4,%
 frame=none%
}


%%ページ番号表示設定
\usepackage{fancyheadings}
\pagestyle{fancy}
\lhead{}
\chead{}
\rhead{\thepage} %右上にページ番号を表示
\cfoot{}
\renewcommand{\headrulewidth}{0pt} %下線を表示しない

%%参考文献名の変更
\renewcommand{\refname}{7\hspace{3.3mm}参考文献}

%%参考文献番号を右肩につける
\makeatletter
\def\@cite#1{\textsuperscript{(#1)}}
\makeatother



