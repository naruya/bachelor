\chapter{実験}
\label{chap:experiment}
本研究では,第\ref{chap:proposal}章で述べた提案手法の有効性を検証するために, で使用されているShepard Metzlerデータセットを用いて,評価実験を行った.本章では,行なった実験の内容について説明した後,実験結果について述べていく.最後に実験結果を踏まえた考察を行う.

\section{実験概要}
adam
\section{データセット}

\section{実験条件}

\section{評価指標}
評価指標には,予測誤差(負の対数尤度),hogehogeスコアを用いる
\subsection{hogehogeスコア}
オリジナル/ノンオリジナル


\section{学習時のテクニック}

\section{実験結果}

\subsection{定量評価(尤度)}

安定した.

潜在変数のサンプリングが安定し,KL項が安定して下がりやすくなったからだと思ふ

尤度が下がった

\subsection{定量評価(hogehogeスコア)}
\subsection{定性評価}
きれいになってくれ