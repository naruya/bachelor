\chapter{考察}
\label{chap:discussion}

本研究では,生成クエリネットワークについて,メタ学習というフレームワークを用いて確率モデルの観点から検証を行い,その議論を踏まえて改善手法を提案し,評価実験を通してその有効性を検証した.本章ではここまでの内容を本研究の貢献という形で整理し,今後の課題や本研究の社会応用の可能性などについて考察を行う.最後に,世界モデル研究としての本研究の位置付けを確認し,今後の展望について述べる.

\section{本研究の貢献}
本研究では,Eslamiらが提案した生成クエリネットワークが持つ課題として,計算機リソースと時間の両面においてかかるコストが大きすぎる点,学習がハイパーパラメータに対して敏感で不安定である点を挙げ,それらの原因として,モデルを構成するニューラルネットワークのアーキテクチャ構築において前提とする確率モデルの設計が曖昧であり,シーン表現や潜在変数として学習されるものに対して理論的な裏付けがなされていない点を指摘した.そこで,生成クエリネットワークが対象とするタスクがメタ学習という機械学習の枠組みに合致している点に注目し,メタ学習の観点から生成クエリネットワークの確率モデルを再度検証した.すると,前述した課題に関係していると考えられる2つの問題点が確率モデルの設計の中に存在することがわかった.1つはシーン表現$\bm{r}$の推論を決定論的に行っている点,もう1つは観測不可能な変数を2つ定義していることに起因する冗長な対数尤度近似が起きてしまっている点である.そこで,本研究ではこれらの問題点を改善する手法を提案した.提案手法では,ベースライン手法である生成クエリネットワークにおいて定義している2つの変数$\bm{r}$と$\bm{z}$を統合し,1つの確率的な潜在変数$\bm{z}$とすることで,既存手法の冗長性を排除し,潜在変数の確率的な推論を行うことを可能にした.これにより,提案手法がベースライン手法と比較して,安定かつ高速に学習可能となり,実験を通してその有効性が示された.また,より簡潔なモデルとなったことで,学習対象となるパラメータの数をベースライン手法の4割以下まで削減することに成功し,実世界応用が現実的な領域まで改善させることができた.

\section{今後の課題}
\subsection{実環境での検証}
本研究では,シミュレータを用いて作成された多種多様な環境の大量の観測画像データを用いて,生成クエリネットワークをより安定かつ高精度に学習するための手法を提案した.今後は実世界での応用を見据えて,現実世界で集めたデータを用いての実験を行っていく必要がある.特に,実世界では視点座標を計測するセンサーにはノイズが含まれることが想定されるため,ノイズが提案手法の精度にどの程度影響を与えるかなどについて検証を行うことが求められる.また,実環境ではシミュレータを用いた場合のように大量のデータを収集することが難しいと考えられるため,シミュレータで作成したデータセットを用いて学習したモデルを少量の実世界データに対して転移学習させるといった方向性も検討していくべきである.

\subsection{時間変化する環境への対応}
本研究では,環境は常に一定であることを前提としていたが,実世界の環境は時間によって変化しており,現状のモデルではそのような環境に対しては対応することができない.そのため,時系列遷移を考慮したような発展手法の考案も今後の研究テーマの1つとして考えられる.具体的には,提案手法においてシーンを表現する潜在変数である$\bm{z_i}$を各時間ステップにおける状態の表現であると考えて,その時系列遷移をモデル化することができれば,時系列を含めた3次元空間のモデルを構築することができると考えられる.このような環境のダイナミクスのモデル化は今後の大きな課題である.

\section{社会応用}
本研究で対象としたような2次元の画像から環境の3次元構造を推定する枠組みは,以前からStructure from Motion(SfM)という技術として,コンピュータビジョンの分野で盛んに研究が行われており,様々な社会応用が考えられる.例えば,拡張現実(AR)の技術では,現実空間の2次元の映像から3次元構造を推定して,その空間上に仮想的な物体を埋め込むような処理を行う必要があるため,環境の空間モデリングが不可欠である.また,一般的にレーザーを用いて記録を行なっていたような3Dスキャナーの技術も画像ベースの手法で置き換えていくことが検討できる.

このような既存の技術を代替するようなものに限らず,本研究で提案している手法はその特性を生かしてこれまで解くことが難しかったようなタスクに対しても,有用な技術になることが予想できる.本研究の手法は,これまでのSfMの技術とは大きく異なり,推定する3次元構造について点群やメッシュのような明示的なモデル化をすることなく,それらを潜在変数のベクトルとして暗黙的に表現している.このような表現は点群などに比べて非常に低次元であり,様々なタスクに転用することが可能であると考えられる.例えば,ロボットが環境中のモノを掴むことを強化学習で学習するような場合,複数のカメラから得られる視点座標と観測画像のセンサー情報から,掴みたいモノの3次元的な構造を低次元な表現として得ることができれば,より効率的に把持タスクを学習することができるようになると予想できる.このように,強化学習などと組み合わさることによって,本研究の手法は様々なタスクへの応用が期待される.

\section{世界モデル}
本研究のように,深層生成モデルを用いて環境のモデルを構築する研究は,{\bf 世界モデル}の研究として近年注目を集めている.これは,限られた観測情報のみから人がラベルづけをすることなしに({\bf 教師なし}で)環境に関する重要な特徴表現を獲得できることが,汎用的な人工知能の実現のために重要であると考えられているからである.例えば,人間の幼児は身の回りにある様々なモノを視覚・聴覚などの複数のセンサーを用いて知覚し,それらと相互作用する中で,自然に世界が3次元的な構造をしていることや,モノが下方向にしか落ちないこと(重力の存在)を理解している.これは,脳の中に実世界を圧縮した分散表現としての世界モデルを学習して,それを元に未来や未知の状態を予測・想像することができるからであり,その予測・想像に基づいて人間は自らが取る行動の計画を立てるといった高度なプランニングを行うことができると考えられる.このような知能の本質を深層生成モデルを用いた世界モデルとして再現する試みは,今後の人工知能研究の発展において非常に重要であり,注力していくべき研究領域である.

本研究では,その一例として,視点と画像という2つのセンサーデータが得られる環境において,視点から画像を予測するようなモデルを扱ったが,今後はこれをさらに発展させていく必要がある.より具体的には,2つのセンサーデータを対等なマルチモーダルデータとして扱い,その同時分布をモデル化することにより,視点から画像の一方向の予測のみではなく,画像から視点への予測も含めた双方向な予測モデルを構築することが挙げられる.

また,画像や視点以外のセンサーデータを用いた世界モデルの構築も重要である.例えば,物体の把持やピッキングを行うロボットへの応用を検討する上では,ハンドの圧力センサーのデータと組み合わせて学習を行うことで,触覚を考慮した世界モデルを構築し,把持やピッキングをする上で有用な特徴表現を獲得することにつなげることができるだろう.

このように,センサーの観測から環境の本質的な潜在表現を獲得する世界モデルの研究は,様々な領域に応用可能なテーマであり,今後もさらに研究が進展していくことが期待される.