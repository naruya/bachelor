\chapter{まとめ}
\label{chap:conclusion}

本論文では,はじめに について取り上げ, その問題点として である点を挙げた.そして,その問題点の原因として, を指摘し, を行なった.


そこで,本論文では,これらの問題点を解消する改善手法の提案を行なった.提案手法では, することで,  し, を可能にした.

実験では,BAIR Push Datasetを用いてベースライン手法を比較し,定性的・定量的な評価を行なった.提案手法は, においてベースライン手法である を上回る結果となり,提案手法が が実験的にも示された.

最後に考察として,今後の課題や社会応用について議論し,本研究が深層生成モデルを用いて観測情報から環境をモデル化する世界モデルの研究の1つとして位置付けられることを述べ,今後の研究の方向性についての展望を整理した.

本研究は,既存の工学的な技術の代替としてのみならず,より汎用的な人工知能技術の達成のために重要な技術の1つであると考えており,今後も実環境への提案手法の適用や,時系列遷移を考慮したモデルの考案,強化学習技術との融合などを見据えて,人工知能分野の研究の発展に貢献していきたいと考えている.
