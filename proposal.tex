\chapter{状態表現の階層性を考慮することによる深層状態空間モデルの拡張}
\label{chap:proposal}
本章では,生成クエリネットワークのメタ学習としての解釈に関する前章での議論を踏まえて,改善手法の提案を行う.前章で述べた通り,既存手法である生成クエリネットワークはシーン固有の知識をもつ変数$\bm{r_i}$をコンテキスト$D_i$から決定論的に推論しているため,コンテキストから推論可能なシーン表現の不確実性を考慮できていない.また,個別のデータに固有の潜在変数$\bm{z_i^k}$と,その事後分布$p(\bm{z_i^q}|\bm{x_i^q}, \bm{v_i^q}, \bm{r_i}; \theta)$を近似する$q(\bm{z_i^q}|\bm{x_i^q}, \bm{v_i^q}, \bm{r_i}; \phi)$の存在により,最適化すべきパラメータが増大してしまっているという問題点がある.そこで,提案手法では,生成クエリネットワークにおけるシーン表現$\bm{r_i}$と潜在変数$\bm{z_i^k}$を統合して,1つの確率的な潜在変数$\bm{z_i}$とすることで,モデルの冗長性を排除し,シーン表現の確率的な推論を可能にすることを目指す.

はじめに,提案手法で扱う問題設定について整理し,その後提案手法の具体的な説明として,確率モデル・最適化とモデルアーキテクチャについて述べていく.

\section{行動条件付き映像予測の問題設定}
\subsection{問題の定式化}
\subsection{BAIR Push Dataset}
\section{ベースライン}
\subsection{ベースラインにおける課題}
\section{提案手法}