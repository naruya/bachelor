\chapter{序論}
\label{chap:introduction}
\section{本研究の背景}

\subsection{ロボット学習}

多様な環境でさまざまなタスクを実行することのできる,汎用的なロボット(generalist robots)(Escudero Rodrigo et al., 2015; Finn, Yu, et al., 2017) の開発は,ロボット工学の最重要課題の一つである.

\subsection{状態表現学習}
DNNは特徴抽出が得意.表現学習と呼ばれる.時間変化する環境に拡張した表現学習のことで,各時刻の生の観測データやその系列からこの状態表現(特徴)を見つけることが目的となる.

\subsection{映像予測}
上記のような強化学習や実機タスクへの応用から(行動条件付き)映像予測だけを切り取った研究も多い.

提案手法の有効性について定性的・定量的な評価を行う.

最後に実験結果を踏まえて,今後の課題と社会応用について述べる.

\section{本論文の構成}
本論文の構成は以下のとおりである.


% 第\ref{chap:prerequisite}章では,本研究の前提となる深層生成モデルやメタ学習についての知識を述べる.

% 第\ref{chap:meta_gqn}章では,生成クエリネットワークの確率モデルについてメタ学習のフレームワークを用いて考察を行い,問題点を指摘する.

% %第\ref{chap:related}章では,を行う.
% 第\ref{chap:proposal}章では,前章の議論を踏まえて,生成クエリネットワークを改善する提案手法について説明する.

% 第\ref{chap:experiment}章では,提案手法に対する評価実験の結果を示し,実験結果への考察を行う.

% 第\ref{chap:discussion}章では,本研究の今後の課題と具体的な社会応用可能性について議論する.

% 第\ref{chap:conclusion}章を本論文のまとめとする.