\documentclass[a4paper,12pt,oneside,openany,dvipdfmx]{jsbook}
\bibliographystyle{junsrt}
\usepackage{amsmath}
\usepackage{amsfonts}
\usepackage{bm}
\usepackage{graphicx}
\usepackage[hyphens]{url}
\usepackage{algorithm}
% \usepackage[noend]{algpseudocode}
%\usepackage{ascmac}
\usepackage{tikz}
\usetikzlibrary{bayesnet}

\makeindex
\setlength{\textwidth}{\fullwidth}
%\setlength{\textheight}{40\baselineskip}
\addtolength{\textheight}{\topskip}
%\setlength{\voffset}{-0.55in}
\newcommand{\argmax}{\mathop{\rm arg~max}\limits}
\newcommand{\argmin}{\mathop{\rm arg~min}\limits}
\newcommand{\average}[1]{\ensuremath{\langle#1\rangle} }
\renewcommand{\figurename}{Fig. }
\renewcommand{\tablename}{Table }

\title{
    平成31年度\\
    卒業論文\\
    複雑環境下でのロボット学習に向けた\\
    深層状態空間モデルを用いた映像予測
}
\author{
    平成32年2月\\
    指導教員 松尾豊教授\\
    \\
    東京大学工学部システム創成学科\\
    知能社会システムコース\\
    03-180961 近藤生也
}
\date{}
\begin{document}
\maketitle
\pagestyle{plain}
\setlength{\baselineskip}{22truept}
\frontmatter
\chapter*{概要}
\addcontentsline{toc}{chapter}{概要}
\label{chap:abstract}

近年,深層ニューラルネットワーク(DNN)を使用した機械学習手法の発展を背景としてロボットの制御方策を自ら学習させるロボット学習の研究が進んでいる.
ロボット学習において未来の観測,特に自らの行動に対するフィードバックを適切に予測するよう学習することは,環境の遷移モデル(内部モデル)を獲得すること,そしてそれを用いたプランニングや行動結果の予測,安全性の評価などに欠かせない.
時間変化する環境のDNNを用いたモデルとして深層状態空間モデル(Deep State Space Model)があり,これは昨今深層強化学習の分野で取り入れられているが,深層状態空間モデルを用いて映像予測を学習することで内部モデルが獲得できるとされている\cite{Gregor2015}.
しかし現在一般的に使われて深層状態空間モデルとその学習方法は各時刻の状態が低次元のベクトルで表現できるという仮定を暗に置いており,高次元な表現を仮定すると学習が進まず,複雑な環境の予測問題にスケールさせる方法は自明でない.

本研究ではまずこのような問題があることを実験的に指摘した上で,状態の表現に階層性を考えることで深層状態空間モデルにより高次元な状態表現を仮定できるように拡張し,低次元な表現を仮定した場合よりも良い表現が獲得できることを示す.評価実験では様々な物体が置かれた机上でのマニピュレーションを題材にしたデータセットを用いてロボットの行動で条件付けた映像予測を行う.単純な深層状態空間モデルをベースラインに設定し,提案手法の定性的・定量的な有効性を示す.
\setcounter{tocdepth}{2}
\tableofcontents
\listoffigures
\listoftables
\mainmatter
\chapter{序論}
\label{chap:introduction}
\section{本研究の背景}
\subsection{ロボット学習と行動条件付き映像予測}
多様な環境で様々なタスクが遂行可能な汎用的なロボット(generalist robots)の開発はロボット工学の最重要課題の一つである\cite{escudero2015distance}.ロボットハードウェアの低価格化,汎用的なロボットソフトウェアの普及に加え,近年の急速な深層学習技術の発展を受けてロボットの制御方策を自ら学習させる{\bf ロボット学習}の研究が進んでおり,ロボットで遂行可能なタスクは着実に増えてきている.

ロボット学習において,将来予測,特に映像予測を明示的に学習することは,

\begin{itemize}
    \item ロボット自身が映像予測を用いた方策をたてることが可能になる
    \item 映像予測結果を人が評価することでロボットの行動を予め評価できる
\end{itemize}
という大きく二つの点からで重要であると言える.一点目の映像予測を用いた方策の例として,Hafnerら\cite{hafner2019planet}は,強化学習の問題設定において明示的に学習した映像予測モデルを用いることで行動系列をランダムにサンプリングして評価するような簡単なアルゴリズムで効率的なプランニングが可能であることを示した.二点目の映像予測を人が評価する例としてEbertらによる研究\cite{ebert2018visual}では学習した映像予測モデルを用いて,ロボットの操作によって予想される物体の移動の軌跡を確率分布として出力することができ,これを用いて人がロボットの行動の正しさを予め判断することができる.

このようにロボット学習における映像予測は重要であるが,映像予測だけを切り取って研究されることも多い.映像予測の中でも,ロボットの行動の結果として観測される映像を予測する問題設定を{\bf 行動条件付き映像予測}と呼び\cite{oh2015actionconditional}\cite{finn2016unsupervised}\cite{babaeizadeh2017stochastic},様々な研究がなされてきている.近年高精度な行動条件付き映像予測手法がいくつか提案されており,ロボット学習研究で扱うタスクの高度化を背景にしてこれらの映像予測手法をより複雑な問題設定に対して適用していきたいと考えられているが,いくつかの研究で既存の行動条件付き映像予測は上手く機能しない可能性があることがわかってきている.ただしここでいう「より複雑な問題設定」とは具体的には環境中に複数の操作対象の物体が隣接し合って置かれている場合,操作対象が布などの非剛体物である場合など,観測の時間変化に多数のパラメータが関与していたり,複雑な物理法則がはたらいているような問題設定を想定している.

\subsection{既存の行動条件付き映像予測手法の問題点}

行動条件付き映像予測手法は大きく{\bf 回帰型ニューラルネットワーク(RNN)ベースの手法}と{\bf 深層状態空間モデル(DSSM)ベースの手法}に分けられる.Hafnerら\cite{hafner2019planet}の研究など深層強化学習の問題で映像予測を明示的に行う場合は後者のDSSMベースの手法が多く採用されるが,映像予測の問題では前者のRNNベースの手法が多く使われている\cite{denton2018stochastic}\cite{villegas2019high}.

RNNベースの手法は予測した1ステップ先の画像を入力にして更に1ステップ先の画像を出力するというような,自らの出力を逐次入力する構造を持つ.RNNベースの手法はDSSMと比較して高精度な映像を生成に長けている反面,近年提案されているRNNベースの手法には (i) 誤差が蓄積しやすい (ii) 文脈を必要とする という問題点がある.

(i) について,RNNベースのモデルを用いると常に直前のフレームを参照して次のフレームを予測するため短い期間の予測であれば精度は高くなるが,予測誤差が蓄積していくために長期の予測には向かないことが示されている\cite{hafner2019planet}.

(ii) について,RNNはモデルの内部状態を十分に更新した後でないと適切に予測が行えず,予測を始める前に{\bf 文脈}としてそれより前の数フレームを与える必要があり,この文脈として与えられるフレーム数が少ないと予測が悪化することが知られている\cite{villegas2019high}.ロボット実機への応用を考えた場合,現在の状態から未来を予測する際に文脈を得るために先に数ステップ行動することは,予測してから行動するという目的意識に反しており実用的できでない.このためRNNベースの手法をロボット実機に応用する際には,文脈として現在の観測という1フレームのみ与れば十分機能するように改善する必要がある.
このように,RNNベースの手法は制約がありそもそも実ロボットへの応用に向いていない可能性がある.

一方,DSSMは各時刻の状態をベクトル(状態ベクトル)で表現し,毎時刻ロボットの行動によって状態ベクトルが遷移し,その時刻に観測される画像は状態ベクトルからの写像であると考えて遷移モデルと写像のモデルを学習する.DSSMは強化学習の分野で長期の予測にも用いられているなど安定した未来の予測に長けているがRNNと比較して画像の生成時に直前の画像を用いないことから高精度な生成は難しく,また映像生成自体を目的にしてDSSMを用いた研究は現状少ない.

\section{本研究の目的}

これらの研究背景を踏まえ,行動条件付き映像予測の問題をより複雑な問題設定にスケールさせることを目指し,本研究では特に実ロボットへの応用を重要視してDSSMベースの行動条件付き映像予測に取り組む.まずDSSMでは単純にモデルを大きくするだけでは映像予測の精度があがらずむしろ悪化することを実験的に示しその理由を簡単に考察する.その上で,モデルが獲得できる情報量を徐々に多くするようなDSSMの拡張方法を提案し,大きなモデルでも学習が進むように改善・予測精度の向上を図る.さらに行動条件付き映像予測用のデータセットを用いて提案手法の有効性についての定性的・定量的な評価を行い,DSSMを使った際にもより高精度な映像予測を可能にすることを目指す.

\section{本論文の構成}

本論文の構成は以下の通りである.

第\ref{chap:prerequisite}章では,前提知識について説明する.

第\ref{chap:settings}章では,本研究で扱う問題設定を整理する.

第\ref{chap:baseline}章では,深層状態空間モデルの問題点について述べる.

第\ref{chap:proposal}章では,状態表現の階層性を考慮することによる深層状態空間モデルの拡張を提案する.

第\ref{chap:experiment}章では,実験を行い提案手法の有効性を示す.

第\ref{chap:discussion}章では,前章までの議論を踏まえて考察し,応用の可能性について述べる.

最後に第\ref{chap:conclusion}章で結論を述べる.
% \chapter{前提知識}
\label{chap:prerequisite}
本章では,まず深層状態空間モデル(Deep State Space Model, 以下DSSM)のベースとなる変分自己符号化器(Variational Auto Encoder, 以下VAE)について説明し,続いてDSSMの説明を行う

\section{変分自己符号化器(VAE)}
\label{section:VAE}

\begin{figure}[tbp]
\begin{center}
  \begin{tikzpicture}[scale=1, transform shape]
    \node[obs] (x1) {$\mathbf{x}$};
    \node[latent, above=of x1] (z1) {$\mathbf{z}$};
    \edge {z1} {x1};
    \end{tikzpicture}
\caption{VAEのグラフィカルモデル}
\label{fig:gm_vae}
\end{center}
\end{figure}

{\bf 変分自己符号化器}({\bf Variational Auto-encoder}, 以下{\bf VAE})\cite{vae}は,深層生成モデルの一種である.
VAEでは,データ$\bm{x} \in \mathbb{R}^n$はある潜在変数$\bm{z} \in \mathbb{R}^m$から生成されると考え,その確率的生成過程$p(\bm{x}|\bm{z})$を多層ニューラルネットワークを用いてモデル化する.
つまり,Fig. \ref{fig:gm_vae}のようなグラフィカルモデルで表現される確率モデルを仮定し,そのパラメータ$\theta$をニューラルネットワークのパラメータとして表現する.
データ$\bm{x}$は手書き数字の画像に該当するが,これはそれぞれの要素が1つのピクセルの値に相当する$28\times28=784$次元のベクトルで表された高次元な表現である.
潜在変数$\bm{z}$は,データ$\bm{x}$をより低次元に表現する.
これは,画像のような高次元なデータは画像空間上の非常に限られた領域に局所的に存在しており,それらはより低次元に表現可能であるとする多様体仮説に基づいている.

すると,データの分布$p(\bm{x})$は,$\theta$によってパラメータ化された条件付き分布$p(\bm{x}| \bm{z}; \theta)$を用いて,
\begin{equation}
p(\bm{x}) = \int p(\bm{x}|\bm{z};\theta) p(\bm{z}) d\bm{z} \label{eq:vae}
\end{equation}
と書くことができる.
%ここで,$p(\bm{x}| \bm{z}; \theta)$は潜在変数からデータを生成するため,デコーダと呼ばれる.
VAEでは,潜在変数の分布について,以下の2つの仮定を置く.
\begin{eqnarray}
p(\bm{z}) &=& \mathcal{N}(\bm{z}|0,\bm{I}) \label{eq:z}\\
p(\bm{z}|\bm{x}) &=& \mathcal{N}(\bm{z}|\mu(\bm{x}),\sigma(\bm{x}))	\label{eq:z_cond}
\end{eqnarray}
式(\ref{eq:z})は,潜在空間が標準正規分布に従っているという仮定であり,式(\ref{eq:z_cond})は,$\bm{x}$に条件づけられた潜在変数の分布が正規分布に従うという仮定である.
ベイズ統計では$p(\bm{z})$は事前分布と呼ばれ,データ$\bm{x}$を観測した後の分布$p(\bm{z}|\bm{x})$は事後分布と呼ばれる.
%また,潜在変数を観測データから推定することを,推論(inference)と呼ぶ.
%$p(\bm{z}|\bm{x})$は,データを低次元の潜在空間に埋め込むため,エンコーダと呼ばれる.
$p(\bm{z}|\bm{x})$を解析的に求めることができるケースは非常に限られており,これが解けない場合,近似分布$q(\bm{z}|\bm{x})$を導入して$p(\bm{z}|\bm{x})$を近似することがベイズ統計ではよく行われる.
VAEでも,$p(\bm{z}| \bm{x})$を別のパラメータ$\phi$を用いて$q(\bm{z}|\bm{x};\phi)$によって近似する.
$p(\bm{z}|\bm{x})$はガウス分布を仮定しているため,$q(\bm{z}|\bm{x};\phi)$もガウス分布を仮定する.
VAEでは,この近似分布$q(\bm{z}|\bm{x};\phi)$もニューラルネットワークを用いて表現する.
%VAEはエンコーダとデコーダの両者をニューラルネットワークで定義する.

\begin{figure}[tbp]
  \begin{center}
    \begin{tikzpicture}[scale=1, transform shape]
      \node[obs] (x1) {$\mathbf{x}$};
      \node[latent, above=of x1] (z1) {$\mathbf{z}$};
      \draw (x1) edge[out=135,in=225,->,dashed] (z1);
      \edge {z1} {x1};
      \end{tikzpicture}
  \caption{推論分布を導入したVAEのグラフィカルモデル.実線は生成モデル,点線は推論モデルを示す.}
  \label{fig:gm_vae_inference}
  \end{center}
  \end{figure}


VAEの目的はデータの分布$p(\bm{x})$を推定することであるため,目的関数は式(\ref{eq:vae})の尤度を最大化することである.
しかし,式(\ref{eq:vae})は$\bm{z}$の周辺化を含み,これを解析的に求めることは困難であるため,尤度そのものを計算することはできない.
%そこで,$\theta$によってパラメータ化されたデコーダ$p(\bm{x}|\bm{z}; \theta)$と,$\phi$によってパラメータ化されたエンコーダ$q(\bm{z}|\bm{x};\phi)$を用いて,式(\ref{eq:vae})の対数をとって,その変分下限を次のように導出する.
そこで,式(\ref{eq:vae})に先ほど定義した$p(\bm{z}|\bm{x})$の近似分布$q(\bm{z}|\bm{x};\phi)$を導入し,その対数をとることで,以下のような変分下限を導出する.
\begin{eqnarray}
\log p(\bm{x}) &=& \log \int p(\bm{x}|\bm{z}; \theta) p(\bm{z}) d\bm{z} \nonumber \\
&=& \log \int q(\bm{z}|\bm{x}; \phi) \frac{p(\bm{x}|\bm{z}; \theta) p(\bm{z})}{q(\bm{z}|\bm{x}; \phi)} d\bm{z} \nonumber \\
&\geq& \int q(\bm{z}|\bm{x}; \phi) \log \frac{p(\bm{x}|\bm{z}; \theta) p(\bm{z})}{q(\bm{z}|\bm{x}; \phi)} d\bm{z} \label{eq:jensen}\\
&=& \int q(\bm{z}|\bm{x}; \phi) \log p(\bm{x}|\bm{z}; \theta) d\bm{z} - \int q(\bm{z}|\bm{x}; \phi) \log \frac{q(\bm{z}|\bm{x}; \phi)}{p(\bm{z})} d\bm{z} \nonumber \\
&=& \mathbb{E}_{\bm{z} \sim q(\bm{z}|\bm{x}; \phi)} [\log p(\bm{x}|\bm{z}; \theta)] - \mathrm{D_{KL}}(q(\bm{z}|\bm{x}; \phi) \| p(\bm{z})) \label{eq:elbo}
\end{eqnarray}
ここで,式(\ref{eq:jensen})でイェンセンの不等式(Jensen's inequality)を用いている.
式(\ref{eq:elbo})の第2項の$\mathrm{D_{KL}}$(カルバックライブラー距離)は,いま$p(\bm{z})$,$q(\bm{z}|\bm{x}; \phi)$共にガウス分布を仮定しているため,解析的に計算することができる.
一方,式(\ref{eq:elbo})の第1項は,解析的には計算できないため,$q(z|x; \phi)$からサンプリングされる$L$個の$\bm{z}$を用いて$\frac{1}{L} \sum_{l} \log p(\bm{x}|\bm{z})$でモンテカルロ近似する.通常,$p(\bm{x}|\bm{z})$はベルヌーイ分布や正規分布とし,その尤度を計算する.
VAEではこの変分下限を目的関数として最大化するように,誤差逆伝播法でニューラルネットワークのパラメータ$\theta$と$\phi$の最適化を行う.
しかし,ここで$q(\bm{z}|\bm{x}; \phi)$からサンプリングを行う部分で勾配の計算ができず,計算グラフが途切れてしまって,誤差逆伝搬法を用いた最適化ができなくなってしまうという問題が生じる.
これを解決するために,VAEでは{\bf 再パラメータ化トリック}({\bf reparameterization trick})を使用する.
%Fig. \ref{fig:reparam}に,再パラメータ化トリックの概略を示す.
通常のサンプリングを行う場合,正規分布$q(\bm{z}|\bm{x}; \phi)$の母数である平均$\mu (\bm{x})$と標準偏差$\sigma (\bm{x})$がニューラルネットワークによって出力された後,$N(\mu (\bm{x}), \sigma (\bm{x}))$から$\bm{z}$をサンプリングするが,これでは計算グラフが途切れてしまう.
そこで,再パラメータ化トリックでは,$\mathcal{N}(\mu (\bm{x}), \sigma (\bm{x}))$から直接サンプリングを行う代わりに,標準正規分布$\mathcal{N}(\bm{0},\bm{I})$からサンプルされる変数$\epsilon$を用いて,$\bm{z}=\mu (\bm{x})+ \epsilon \cdot \sigma (\bm{x})$と計算することによって,計算グラフを途切らせることなく,確率的なサンプリングを可能にする.

\begin{figure}[tbp]
  \begin{center}
    \includegraphics[width=\linewidth]{./figures/vae.png}
    \caption{VAEを用いて生成された画像の例(\cite{vae}より引用)}
    \label{fig:vae}
  \end{center}
\end{figure}

\section{深層状態空間モデル(DSSM)}
\label{section:DSSM}

Non-linear State Space Model, Deep Karman Filters, Deep Markov Model
% \chapter{生成クエリネットワークのメタ学習としての考察}
\label{chap:meta_gqn}
本章では,生成クエリネットワークについて,メタ学習という観点から理論的な考察を行い,その問題点を指摘する.具体的には\ref{section:meta_learning}節で述べたメタ学習の確率モデルとしての定式化に生成クエリネットワークを当てはめて検証し,その中で浮かび上がってきた生成クエリネットワークの問題点をまとめる.


\section{生成クエリネットワークの解釈}
\label{section:interpretation}
生成クエリネットワークで対象としている問題設定はメタ学習の枠組みに当てはめて考えることができる.生成クエリネットワークでは,モノの配置などがそれぞれで異なるようなシーンが多数存在し,あるシーンでの視点座標と観測がいくつか与えられた状況において,別の視点からの観測を予測する.ここでのシーンはメタ学習における個々のデータセットに相当し,モデルはあるシーン(データセット)について,与えられたコンテキスト(訓練データ)を元にクエリ(テストデータ)の視点座標(入力)から観測画像(出力)の写像を学習する.ここでは,シーンごとに入力となる視点座標の確率空間は共通であるが,出力する観測は異なるため,ドメインが共通でタスクが異なる場合のメタ学習の問題設定と一致する.このように考えると,ML-PIPと同様にデータの生成過程をFig. \ref{fig:gm_meta_gqn}のようにモデリングすることができる.ただし,生成クエリネットワークでは,ML-PIPにおける$t$番目のデータセットに固有の知識をもつ変数$\psi^{(t)}$に相当するような変数として,$i$番目のシーンに固有の知識をもつ変数$\bm{r_i}$が存在するが,それに加えて個別のデータ($\bm{v_i^k}, \bm{x_i^k}$)に固有の潜在変数$\bm{z_i^k}$を仮定してモデル化している点が通常のML-PIPの枠組みとは異なる.

\begin{figure}[tbp]
\begin{center}
\begin{tikzpicture}
  % Define nodes
  \node[obs] (x_k) {$\bm{x_i^k}$};
  \node[obs, left=of x_k] (v_k) {$\bm{v_i^k}$};
  \node[latent, right=of x_k] (r) {$\bm{r_i}$};
  \node[latent, above=of v_k] (z_k) {$\bm{z_i^k}$};
   
   \node[obs, right=of r] (x_q) {$\bm{x_i^q}$};
   \node[obs, right=of x_q] (v_q) {$\bm{v_i^q}$};
   \node[latent, above=of v_q] (z_q) {$\bm{z_i^q}$};
   
   % Plates
  \plate {query} {(v_q)(x_q)(z_q)} {クエリ} ;
  \plate {context} {(v_k)(x_k)(z_k)} {$D_i(k=1,...,M)$} ;
  \plate {scene} {(context)(query)(r)} {シーン$i$} ;

   
   \node[const, above=of scene] (theta) {$\theta$};

  % Connect the nodes  
  \edge {v_k} {x_k};
  \edge {v_q} {x_q};
  \edge {r} {x_q};
  \edge {r} {x_k};
  \edge {r} {z_k};
  \edge {r} {z_q};
  \edge {z_k} {x_k};
  \edge {z_q} {x_q};
  \edge {v_k} {z_k};
  \edge {v_q} {z_q};
  
  \edge [loop above] {z_k} {z_k} ; 
  \edge [loop above] {z_q} {z_q} ; 
  
  \edge {theta} {x_k};
  \edge {theta} {x_q};
  \edge {theta} {r};
  \edge {theta} {z_k};
  \edge {theta} {z_q};


\end{tikzpicture}
\caption{メタ学習としての生成クエリネットワークのグラフィカルモデル}
\label{fig:gm_meta_gqn}
\end{center}
\end{figure}

このように生成過程をモデル化すると,コンテキスト$D_i$から$\bm{r}$を出力する表現ネットワークは,メタ学習において分布$p( \bm{r_i} | D_i ; \theta )$を近似する$q^\prime ( \bm{r_i} | D_i ; {\phi}^\prime )$の役割を果たしていると解釈することができる.すると,生成クエリネットワークで最大化する目的関数は以下のようになり,式(\ref{eq:gqn_draw_elbo})と一致する.
\begin{eqnarray}
\log p ( \bm{x _ i  ^ q} | \bm{v _i ^ q} , D_i ; \theta ) 
&\simeq& \log \int p ( \bm{x _i ^ q} | \bm{v _i ^ q} , \bm{r _ i} ; \theta ) q^\prime ( \bm{r_i} | D_i ; {\phi}^\prime ) \mathrm { d } \bm{r_i} \label{eq:approximation_r} \\
&=& \log p ( \bm{x_i^q} | \bm{v_i^q}, \bm{r_i}; \theta)  \quad (\bm{r_i} = f(D_i; {\phi}^\prime)) \\
&=& \log \int p ( \bm{x _i^ q} | \bm{v_i ^ q} , \bm{z _i^ q} , \bm{r _i} ; \theta ) \pi (\bm{z_i^q} | \bm{v_i^q}, \bm{r_i}; \theta) \mathrm { d } \bm{z_i^q} \\
&\geq& 
\mathbb{E} _ {\bm{z_i^q} \sim q(\bm{z_i^q}|\bm{x_i^q}, \bm{v_i^q}, \bm{r_i}; \phi)} [\log p ( \bm{x_i ^ q} | \bm{v_i ^ q} , \bm{z _i^ q} , \bm{r_i} ; \theta )]
- \mathrm{D_{KL}} (q||\pi) \\
&=& 
\mathbb{E} _ {\bm{z_i^q} \sim q(\bm{z_i^q}|\bm{x_i^q}, \bm{v_i^q}, \bm{r_i}; \phi)} [\log p ( \bm{x_i ^ q} | \bm{v_i ^ q} , \bm{z _i^ q} , \bm{r_i} ; \theta )
-  \sum _ { l = 1 } ^ { L } {\mathrm { D } _ { \mathrm { KL } }}
	(q_l || \pi_l) ] \label{eq:meta_gqn_elbo}
\end{eqnarray}

これがメタ学習としての生成クエリネットワークの定式化となる.

\section{生成クエリネットワークの問題点}
\label{section:problem}
前節のメタ学習としての生成クエリネットワークの解釈を踏まえると,生成クエリネットワークにはいくつかの問題点があることがわかる.
\begin{description}
\item[$q\prime \left( \bm{r_i} | D_i ; {\phi}^\prime \right)$が決定論的な関数で定義されている]\mbox{}\\
生成クエリネットワークでは,シーン表現$\bm{r_i}$の分布を近似する$q\prime \left( \bm{r_i} | D_i ; {\phi}^\prime \right)$を表現ネットワークを用いて決定論的な関数として定義しているが,事前に与えられたシーンのサンプル$D_i$から得られるシーン固有の知識というのは,常に不確実性を含んでおり,決定論的な関数ではそれを表現することは不可能である.例えば,$D_i$として1つの視点からの観測のみが与えられた場合,物体の影となっていて観測できない部分などの知識はそれのみから得ることは不可能であるため,シーンの全てを表すような表現は決定論的に導くことは一般にできない.生成クエリネットワークでは,その不確実性のモデリングを個別のデータに依存する確率的な潜在変数$\bm{z_i^k}$によって行なっているが,本来これはシーン表現$\bm{r_i}$が担うべき役割であり,さらに観測不可能な変数を増やすことで,複雑で大きなモデルが必要となり,学習にかかるコスト増大の原因になっていると考えられる.
\item[冗長な近似を行なっている]\mbox{}\\
生成クエリネットワークでは,式(\ref{eq:approximation_r})での$q^\prime ( \bm{r_i} | D_i ; {\phi}^\prime )$による$p( \bm{r_i} | D_i ; \theta )$の近似に加えて,式(\ref{eq:meta_gqn_elbo})で$q(\bm{z_i^q}|\bm{x_i^q}, \bm{v_i^q}, \bm{r_i}; \phi)$による$p(\bm{z_i^q}|\bm{x_i^q}, \bm{v_i^q}, \bm{r_i}; \theta)$の近似が発生しており,学習時には3つのパラメータ$\theta, \phi, {\phi}^\prime$を同時に最適化することが必要となっている.生成クエリネットワークの学習の不安定性はこのような冗長な近似による複雑な最適化が主な原因となっていると考えられる.
\end{description}

このように,生成クエリネットワークの課題として挙げられていた学習時の時間面・リソース面のコストの大きさや学習の不安定性は,その確率モデルの設計が原因であるとして説明できることがわかり,これらを踏まえた改善手法の提案が必要である.
%するシーンに固有の知識をもつ変数$\bm{r_i}$と個別のデータ(\bm{$v_i^k}, \bm{x_i^k}$)に固有の潜在変数$\bm{z_i^k}$という2つの観測不可能な変数を扱っているため,とで近似を2回行わなければならない.このような冗長な近似を行うと,近似のためのニューラルネットワークを新たに追加する必要が生まれるため,パラメータ数の増加につながり,最適化が難しくなる原因となっている.

\chapter{状態表現の階層性を考慮することによる深層状態空間モデルの拡張}
\label{chap:proposal}

第\ref{chap:baseline}章の問題を受け,第\ref{chap:proposal}章ではシンプルな帰納バイアスを導入することによってDSSMを拡張する方法を提案する.はじめに本研究で扱う問題設定について改めて整理し,続けて提案手法とその既存の類似手法について述べる.

\section{提案手法}
第\ref{chap:baseline}章で述べたようにベースラインのDSSMでは潜在変数の次元を大きくすると学習がうまくいかなかった.しかし予備実験で得た,低次元の潜在変数を用いたときには部分的に学習が進んだという事実をヒントにし,状態変数の次元を大きくしていく方向性はそのままで状態表現の階層性を考えることにより,より複雑な問題設定においても学習可能なDSSMの拡張方法を提案する.

\begin{figure}[tp]
  \begin{center}
    \includegraphics[width=\linewidth]{./figures/hierarchical.png}
    \caption{獲得される状態表現の包含関係}
    \label{fig:hierarchical}
  \end{center}
\end{figure}

\subsection{状態表現の階層性}
はじめに,ベースラインのDSSMにおいて潜在変数の次元を変えた時に獲得される情報について考察する.低次元の状態変数で獲得できる情報は高次元の状態変数を用いた場合にも当然獲得できると考えた場合,図(\ref{fig:hierarchical})のように高次元の状態変数が持つ情報は低次元の状態変数が持つ情報をほぼ内包していると考えることができる.
ここで状態変数の次元をより大きくしたときに精度がむしろ悪化することが問題であったが,これは第\ref{chap:baseline}章で述べたとおり,状態変数の次元が大きくなったときに高次元ベクトルから高次元ベクトルへの写像を学習する必要が生じあるべき写像先がなかなか定まらないことが原因であると考えられ,何らかの方法で遷移モデルの学習を補助することで図のようにより多くの情報を獲得できる可能性がある.

\subsection{階層的な状態表現の遷移}
ベースラインの状態表現の遷移は図(\ref{fig:transition_base})が示すように状態変数が持つすべての情報を一度に変換することを考えているが,直感的に一括で変換することは学習が難しいと思われる.状態表現を一括で変換する代わりに,前節のような階層性の概念を導入することで図(\ref{fig:transition_proposal})のように簡単に遷移が学習できる部分から順に遷移させていくような方法を考えることができる.

このような階層的な状態表現の遷移を考えると,はじめから高次元の状態表現の遷移を考えずに学習の習熟度に合わせて徐々に高次元の状態ベクトルの遷移を学習することができ,学習がスムースに進みやすくなると考えられる.

\begin{figure}[tbp]
  \begin{center}
    \includegraphics[width=0.5\linewidth]{./figures/transition_base.png}
    \caption{ベースラインの状態遷移の模式図}
    \label{fig:transition_base}
  \end{center}
\end{figure}

\begin{figure}[tbp]
  \begin{center}
    \includegraphics[width=0.8\linewidth]{./figures/transition_proposal.png}
    \caption{提案手法の状態遷移の模式図}
    \label{fig:transition_proposal}
  \end{center}
\end{figure}

% \caption[hoge]{fuga}
\begin{figure}[tbp]
  \begin{center}
    \includegraphics[width=\linewidth]{./figures/proposal.png}
    \caption[提案手法のグラフィカルモデル]{提案手法のグラフィカルモデル.点線の$s^1$, $s^2$の推論分布は簡単のため時刻t, t-1でのみ記載している.左側の図が評価時の生成過程を示しており,右側の図が訓練時のデータの流れを推論分布を含めて示している.右図について,2つずつ記載されている$o_t$は同じデータを示すが.異なるsから独立に生成されることを明示している.}
    \label{fig:proposal}
  \end{center}
\end{figure}

\subsection{確率モデル・最適化}

ここまでで状態変数の階層性とその遷移を考えたが,この階層性の仮定はDSSMの性能の向上に十分寄与しうると考え,状態変数の階層性を帰納バイアスとしてDSSMに組み込み以下のようなモデルとその最適化アルゴリズムを提案する.

\vspace{\baselineskip}
提案手法のグラフィカルモデルを図(\ref{fig:proposal})に示す.提案手法は,DSSMの状態変数をN層に階層化したモデルである.図(\ref{fig:proposal})は2階層の提案モデルを表している.図(\ref{fig:proposal})の上側の状態変数から一階層の状態変数・二階層の状態変数と呼ぶことにすると一階層の状態変数が低次元ベクトル,二階層の状態変数が高次元ベクトルになっており,高次元の状態変数の生成・推論時に低次元の状態変数を用いるようなモデルになっている.高次元の状態変数の遷移時に低階層の状態変数を用いて写像先に関する情報を補助的に与えることで,学習を安定化させる効果が期待される.またモデルの評価時には,最高層の観測の生成モデル $p(o_t|s^N_t)$ を用いる.

\begin{algorithm}[tbp]               
  \caption{N階層DSSMの学習アルゴリズム}
  \label{alg1}
  \begin{algorithmic}
    \REQUIRE 階層数 $N$ 
    \FOR{i = 1 to $N$}
      \WHILE{$i$階層のネットワークの学習が収束していない}
        \STATE $1 \sim i-1$階層のネットワークのパラメータを固定し, 
        \STATE $i$階層のネットワークを以下の目的関数で学習する
        \STATE $L(a_{1:T}, o_{1:T}) = $
        \STATE $ \hspace{2em}\sum_{t=1}^T ( \mathbb{E}_{s^i_t} [\log p(o_t|s^i_t)] - \mathbb{E}_{s^i_{t-1}} [\mathrm{D_{KL}}(q(s^i_t|s^i_{t-1}, a_t, o_t) \| p(s^i_t|s^i_{t-1}, a_t, o_t, s^{i-1}_t))]) $
      \ENDWHILE
    \ENDFOR
  \end{algorithmic}
\end{algorithm}

\vspace{\baselineskip}
次に提案手法の学習アルゴリズムをアルゴリズム\ref{alg1}に示す.この学習アルゴリズムは前節「階層的な状態表現の遷移」で述べた,習熟度に合わせて徐々に高次元の状態ベクトルの遷移を学習するという考え方に基づいており,これにより安定した学習が見込める.今回簡単のために高階層の潜在表現の学習時にはそれより低階層の状態表現の学習を止めているが,他の方法も考えられ,これついては考察「低次元状態ベクトルの階層の再学習」で述べる.



% 確率的生成過程は以下
% \begin{equation}
%   p(o_{1:T}|a_{1:T}) = \prod_{t=1}^T \iint p(o_t|s^2_t) p(s^2_t|s^2_{t-1}, a_t, s^1_t) p(s^1_t|s^1_{t-1}, a_t) d{s^1_t}{ds^2_t}
% \end{equation}


% この時の変分下限は以下
% \begin{eqnarray}
%   \ (ELBO) \nonumber \\
%   &=& \sum_{t=1}^T \left( \mathbb{E}_{s^2_t \sim q(s^2_t|a_{1:t}, o_{1:t})} [\log p(o_t|s^2_t)] \right. \nonumber \\
%   && \hspace{2em} \left. - \mathbb{E}_{s^1_{t-1} \sim q(s^1_{t-1}|a_{1:t-1}, o_{1:t-1}), 改行したい s^2 \sim}  [\mathrm{D_{KL}}(q(s_t|s_{t-1}, a_t, o_t) \| p(s_t|s_{t-1}, a_t, o_t))] \right. \nonumber \\
%   && \hspace{2em} \left. - \mathbb{E}_{s^1_{t-1} \sim q(s^1_{t-1}|a_{1:t-1}, o_{1:t-1}), 改行したい s^2 \sim } [\mathrm{D_{KL}}(q(s_t|s_{t-1}, a_t, o_t) \| p(s_t|s_{t-1}, a_t, o_t))] \right) \nonumber \\
%   \label{eq:hssm_elbo}
% \end{eqnarray}


% 目的関数は以下
% % 
% \begin{eqnarray}
%   \ (目的関数) \nonumber \\
%   &=& \sum_{t=1}^T \left( \mathbb{E}_{s^2_t \sim q(s^2_t|a_{1:t}, o_{1:t})} [\log p(o_t|s^2_t)] + \beta \mathbb{E}_{s^1_t \sim q(s^1_t|a_{1:t}, o_{1:t})} [\log p(o_t|s^1_t)] \right. \nonumber \\
%   && \hspace{2em} \left. - \mathbb{E}_{s^1_{t-1} \sim q(s^1_{t-1}|a_{1:t-1}, o_{1:t-1}), 改行したい s^2 \sim}  [\mathrm{D_{KL}}(q(s_t|s_{t-1}, a_t, o_t) \| p(s_t|s_{t-1}, a_t, o_t))] \right. \nonumber \\
%   && \hspace{2em} \left. - \mathbb{E}_{s^1_{t-1} \sim q(s^1_{t-1}|a_{1:t-1}, o_{1:t-1}), 改行したい s^2 \sim } [\mathrm{D_{KL}}(q(s_t|s_{t-1}, a_t, o_t) \| p(s_t|s_{t-1}, a_t, o_t))] \right) \nonumber \\
%   \label{eq:hssm_loss}
% \end{eqnarray}

\section{類似手法との差分}
提案手法と類似手法の差分について整理する.DSSM自体を用いた映像予測の研究はこれまであまりされていないが,これは第\ref{chap:introduction}章でも述べたとおりDSSMは自己回帰モデルと比べて高精度な生成には向いていないためだと考えられる.そのため本節では階層性を考慮した画像生成と映像生成の既存研究について取り上げる.

まず,モデルに階層性を取り入れた画像生成の先行研究としてDRAW\cite{gregor2015draw},PGGAN\cite{karras2017progressive}がある.DRAWはVAEベースの深層生成モデルであり,潜在変数を複数用意し階層的にすることでデータの潜在表現に単純な正規分布を仮定しないより複雑な表現を可能にしている.提案手法はDRAWを時間変化するデータを扱う問題設定に拡張したようなモデルとなっているが,DRAWは一度にすべての潜在表現を学習するのに対し,時系列データを扱う提案手法では毎時刻の状態表現の遷移を学習することが難しいため潜在表現を順に学習するアルゴリズムを採用している.VAEベースで潜在表現の階層性を取り入れた研究としては他にも\cite{snderby2016ladder}\cite{zhao2017learning}\cite{maale2019biva}がある.PGGANは敵対的生成モデルを用いた画像生成手法で,低解像度の画像の生成からはじめ,学習が進むにつれてニューラルネットワークモデルを徐々に多層にすることで高解像度の画像生成を行う.低次元の隠れ変数を変数から学習していくという点で提案手法と似ているが,PGGANは何も条件付けしない生成を行っており,過去の状態や行動で条件付けして画像を生成する行動条件付き映像予測の問題設定に用いる方法は自明ではなかった.

次に,映像予測で階層的なモデルを扱う手法としFutureGAN\cite{Aigner_2019},てVRNN\cite{castrejon2019improved}がある.FutureGANはPGGANを映像予測に応用した手法で,過去の6フレーム程度の映像を入力とし未来の25フレーム程度の映像を予測することができる.VRNNは直前のフレームを用いて次のフレームを予測する自己回帰モデルになっており,潜在変数から画像を出力する過程で得られるいくつかの隠れ変数を次のフレームの予測にも用いるようにして潜在変数の階層性を考えたモデルになっている.FutureGANやVRNNは予測する際に過去数フレームの映像を用意する必要があるが,ベースラインのDSSMや提案手法では初期状態の推論用に過去1フレームのみ使うことを前提としたモデルとなっている.

% あれaction conditionlな先行研究って,visual forsightとか,SV2Pとかしかなかったっけ?

% videoflowはaction conditionalじゃないしー


% \chapter{実験}
\label{chap:experiment}
本研究では,第n章で述べた提案手法の有効性を検証するために,を用いて,評価実験を行った.本章では,行なった実験の内容について説明した後,実験結果について述べていく.最後に実験結果を踏まえた考察を行う.

\section{実験内容}
潜在変数の次元は,5章で示すように学習がある程度進むことが確認できた〇〇64次元~〇〇次元をベースラインとして設定する.

\subsection{実験概要}
adam
評価指標には,予測誤差(負の対数尤度),hogehogeスコアを用いる
push datasetを使う
本実験の

\subsection{BAIR Push Dataset}
BAIR Push Datasetは,行動条件付き映像予測,行動条件をつけない映像予測のどちらでも用いられるデータセットであり,によって制作され公開されている.
データセットの内容のうち今回用いるのは,様々な物体がおかれた机の上をアームロボットがランダムに掻き乱すようにして集められた行動系列$\vec{a}$と固定視点から観測された画像系列$\vec{o}$のセット$\{\vec{a}, \vec{o}\}$である.
行動系列$\vec{a}$には,具体的にはロボットのエンドエフェクタの目標位置?
データは10hzで撮られている.

\subsection{hogehogeスコア}
オリジナル/ノンオリジナル
\subsection{モデルアーキテクチャ}
\subsection{}
\subsection{実装}
pytorch + Pixyz, tensorflow dataset
学習時のテクニック
\begin{itemize}
    \item リパらトリック
    \item min stddev
    \item share
\end{itemize}

\section{実験結果}
\subsection{定量評価(尤度)}

安定した.

潜在変数のサンプリングが安定し,KL項が安定して下がりやすくなったからだと思ふ

尤度が下がった

\subsection{定量評価(hogehogeスコア)}
\subsection{定性評価}
きれいになってくれ
% \chapter{考察}
\label{chap:discussion}

\section{本研究の貢献}

\section{今後の課題}

\section{社会応用}

\section{世界モデル}
\chapter{まとめ}
\label{chap:conclusion}

本論文では,はじめに について取り上げ, その問題点として である点を挙げた.そして,その問題点の原因として, を指摘し, を行なった.


そこで,本論文では,これらの問題点を解消する改善手法の提案を行なった.提案手法では, することで,  し, を可能にした.

実験では,BAIR Push Datasetを用いてベースライン手法を比較し,定性的・定量的な評価を行なった.提案手法は, においてベースライン手法である を上回る結果となり,提案手法が が実験的にも示された.

最後に考察として,今後の課題や社会応用について議論し,本研究が深層生成モデルを用いて観測情報から環境をモデル化する世界モデルの研究の1つとして位置付けられることを述べ,今後の研究の方向性についての展望を整理した.

本研究は,既存の工学的な技術の代替としてのみならず,より汎用的な人工知能技術の達成のために重要な技術の1つであると考えており,今後も実環境への提案手法の適用や,時系列遷移を考慮したモデルの考案,強化学習技術との融合などを見据えて,人工知能分野の研究の発展に貢献していきたいと考えている.

\appendix
%\chapter*{付録}
\addcontentsline{toc}{chapter}{付録}
\label{chap:appendix}

ふろく
ふろく
ふろく

\chapter*{謝辞}
\addcontentsline{toc}{chapter}{謝辞}
本論文を作成するにあたり,多くの方々にご協力をいただきました.

最後に,配属からの一年間,あらゆる面でサポートをしてくださった松尾研究室の皆様に御礼申しあげて,謝辞とさせて頂きます.


\begin{flushright}
東京大学工学工学部システム創成学科\\
知能社会システムコース\\
松尾研究室学部 4年\\
平成30年2月 谷口尚平\\
\end{flushright}
\bibliography{references}
\end{document}
