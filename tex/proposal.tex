\chapter{状態表現の階層性を考慮することによる深層状態空間モデルの拡張}
\label{chap:proposal}

第三章の問題を受け,第四章ではシンプルな帰納バイアスを導入することによってDSSMを拡張する方法を提案する.はじめに本研究で扱う問題設定について改めて整理し,続けて提案手法とその既存の類似手法について述べる.

\section{問題設定の整理}

本研究では行動条件付き映像予測の問題を解く.具体的には,ある行動主体が実行した行動系列$a_{1:10}$と初期観測$o_0$が与えられたときに$o_{1:10}$を生成し,その生成される映像の尤度を高めることを目指す.ただし訓練時には行動系列と観測系列の組$\{\vec{a}, \vec{o}\}$の訓練用のデータセットを用いることができ,評価時には,訓練データには含まれないが訓練データと同じ条件で収集された評価用のデータセットを用いる.
本研究の目的は,この条件付き映像予測問題においてベースラインにDSSMを設定し,DSSMをより複雑な環境での映像予測問題にもスケールできるように拡張することである.

\section{提案手法}
三章で述べたようにベースラインのDSSMでは潜在変数の次元を大きくすると学習がうまくいかなかった.しかし予備実験で得た,低次元の潜在変数を用いたときには部分的に学習が進んだという事実をヒントにし,状態変数の次元を大きくしていく方向性はそのままで状態表現の階層性を考えることにより,より複雑な問題設定においても学習可能なDSSMの拡張方法を提案する.

\begin{figure}[tp]
  \begin{center}
    \includegraphics[width=\linewidth]{./figures/hierarchical.png}
    \caption{hierarchical}
    \label{fig:hierarchical}
  \end{center}
\end{figure}

\subsection{状態表現の階層性}
はじめに,ベースラインのDSSMにおいて潜在変数の次元を変えた時に獲得される情報について考察する.低次元の状態変数で獲得できる情報は高次元の状態変数を用いた場合にも当然獲得できると考えた場合,図(\ref{fig:hierarchical})のように高次元の状態変数が持つ情報は低次元の状態変数が持つ情報をほぼ内包していると考えることができる.
ここで状態変数の次元をより大きくしたときに精度がむしろ悪化することが問題であったが,これは三章で述べたとおり,状態変数の次元が大きくなったときに高次元ベクトルから高次元ベクトルへの写像を学習する必要が生じあるべき写像先がなかなか定まらないことが原因であると考えられ,何らかの方法で遷移モデルの学習を補助することで図のようにより多くの情報を獲得できる可能性がある.

\subsection{階層的な状態表現の遷移}
ベースラインの状態表現の遷移は図(\ref{fig:transition_base})が示すように状態変数が持つすべての情報を一度に変換することを考えているが,直感的に一括で変換することは学習が難しいと思われる.状態表現を一括で変換する代わりに,前節のような階層性の概念を導入することで図(\ref{fig:transition_proposal})のように簡単に遷移が学習できる部分から順に遷移させていくような方法を考えることができる.

このような階層的な状態表現の遷移を考えると,はじめから高次元の状態表現の遷移を考えずに学習の習熟度に合わせて徐々に高次元の状態ベクトルの遷移を学習することができ,学習がスムースに進みやすくなると考えられる.

\begin{figure}[tbp]
  \begin{center}
    \includegraphics[width=0.5\linewidth]{./figures/transition_base.png}
    \caption{transition base}
    \label{fig:transition_base}
  \end{center}
\end{figure}

\begin{figure}[tbp]
  \begin{center}
    \includegraphics[width=0.8\linewidth]{./figures/transition_proposal.png}
    \caption{transition proposal}
    \label{fig:transition_proposal}
  \end{center}
\end{figure}

% \caption[hoge]{fuga}
\begin{figure}[tbp]
  \begin{center}
    \includegraphics[width=\linewidth]{./figures/proposal.png}
    \caption[提案手法のグラフィカルモデル]{提案手法のグラフィカルモデル.点線の$s^1$, $s^2$の推論分布は簡単のため時刻t, t-1でのみ記載している.proposal (学習時) 2つずつ記載されている$o_t$は同じデータを示すが.異なるsから独立に生成されることを明示している.}
    \label{fig:proposal}
  \end{center}
\end{figure}

\subsection{確率モデル・最適化}

ここまでで状態変数の階層性とその遷移を考えたが,この階層性の仮定はDSSMの性能の向上に十分寄与しうると考え,状態変数の階層性を帰納バイアスとしてDSSMに組み込み以下のようなモデルとその最適化アルゴリズムを提案する.

\vspace{\baselineskip}
提案手法のグラフィカルモデルを図(\ref{fig:proposal})に示す.(TODO: 三層以上の場合も図にする),
提案手法は,DSSMの状態変数をN層に階層化したモデルである.図(\ref{fig:proposal})は2階層の提案モデルを図(\ref{fig:proposal_big} TODO)は3階層以上の提案モデルを表している.図(\ref{fig:proposal})の上側の状態変数から一階層の状態変数・二階層の状態変数と呼ぶことにすると一階層の状態変数が低次元ベクトル,二階層の状態変数が高次元ベクトルになっており,高次元の状態変数の生成・推論時に低次元の状態変数を用いるようなモデルになっている.高次元の状態変数の遷移時に低階層の状態変数を用いて写像先に関する情報を補助的に与えることで,学習を安定化させる効果が期待される.

\begin{algorithm}[tbp]               
  \caption{N階層DSSMの学習アルゴリズム TODO: 書き直す}
  \label{alg1}                          
  \begin{algorithmic}
    \FOR{一階層目からN階層目まで}
      \WHILE{学習が収束していない}
        \STATE 現在の階層より上の階層のパラメータを固定し, 
        \STATE 現在の階層を以下の目的関数で学習する
        \STATE $L(x)$ = その階層での再構成 + その階層の$KL(q||p)$ \\
      \ENDWHILE
    \ENDFOR
  \end{algorithmic}
\end{algorithm}

\vspace{\baselineskip}
次に提案手法の学習アルゴリズムをアルゴリズム\ref{alg1}に示す.この学習アルゴリズムは前節「階層的な状態表現の遷移」で述べた,習熟度に合わせて徐々に高次元の状態ベクトルの遷移を学習するという考え方に基づいており,これにより安定した学習が見込める.今回簡単のために高階層の潜在表現の学習時にはそれより低階層の状態表現の学習を止めているが,他の方法も考えられ,これついては考察「低次元状態ベクトルの階層の再学習」で述べる.



% 確率的生成過程は以下
% \begin{equation}
%   p(o_{1:T}|a_{1:T}) = \prod_{t=1}^T \iint p(o_t|s^2_t) p(s^2_t|s^2_{t-1}, a_t, s^1_t) p(s^1_t|s^1_{t-1}, a_t) d{s^1_t}{ds^2_t}
% \end{equation}


% この時の変分下限は以下
% \begin{eqnarray}
%   \ (ELBO) \nonumber \\
%   &=& \sum_{t=1}^T \left( \mathbb{E}_{s^2_t \sim q(s^2_t|a_{1:t}, o_{1:t})} [\log p(o_t|s^2_t)] \right. \nonumber \\
%   && \hspace{2em} \left. - \mathbb{E}_{s^1_{t-1} \sim q(s^1_{t-1}|a_{1:t-1}, o_{1:t-1}), 改行したい s^2 \sim}  [\mathrm{D_{KL}}(q(s_t|s_{t-1}, a_t, o_t) \| p(s_t|s_{t-1}, a_t, o_t))] \right. \nonumber \\
%   && \hspace{2em} \left. - \mathbb{E}_{s^1_{t-1} \sim q(s^1_{t-1}|a_{1:t-1}, o_{1:t-1}), 改行したい s^2 \sim } [\mathrm{D_{KL}}(q(s_t|s_{t-1}, a_t, o_t) \| p(s_t|s_{t-1}, a_t, o_t))] \right) \nonumber \\
%   \label{eq:hssm_elbo}
% \end{eqnarray}


% 目的関数は以下
% % 
% \begin{eqnarray}
%   \ (目的関数) \nonumber \\
%   &=& \sum_{t=1}^T \left( \mathbb{E}_{s^2_t \sim q(s^2_t|a_{1:t}, o_{1:t})} [\log p(o_t|s^2_t)] + \beta \mathbb{E}_{s^1_t \sim q(s^1_t|a_{1:t}, o_{1:t})} [\log p(o_t|s^1_t)] \right. \nonumber \\
%   && \hspace{2em} \left. - \mathbb{E}_{s^1_{t-1} \sim q(s^1_{t-1}|a_{1:t-1}, o_{1:t-1}), 改行したい s^2 \sim}  [\mathrm{D_{KL}}(q(s_t|s_{t-1}, a_t, o_t) \| p(s_t|s_{t-1}, a_t, o_t))] \right. \nonumber \\
%   && \hspace{2em} \left. - \mathbb{E}_{s^1_{t-1} \sim q(s^1_{t-1}|a_{1:t-1}, o_{1:t-1}), 改行したい s^2 \sim } [\mathrm{D_{KL}}(q(s_t|s_{t-1}, a_t, o_t) \| p(s_t|s_{t-1}, a_t, o_t))] \right) \nonumber \\
%   \label{eq:hssm_loss}
% \end{eqnarray}

\section{類似手法との差分 TODO}
本節では提案手法と類似手法の差分について整理する.DSSMを映像生成自体に用いた研究は私の知る限りなく,これは1章でも述べたとおり,自己回帰モデルと比べて高精度な生成には向いていないためだと考えられる.そのため本節では階層性を考慮した既存研究について取り上げる.
\begin{itemize}
  \item DRAW[]はVAEの潜在変数に階層性をもたせたモデル画像生成で用いられる
  \item 多層RNN[]は
  \item PGANは
\end{itemize}


% videoflowはaction conditionalじゃないしー

