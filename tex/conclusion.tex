\chapter{結論}
\label{chap:conclusion}

本論文では,実機ロボットへの応用を見据えて深層状態空間モデルを用いた映像予測について取り上げた.まず深層強化学習などで用いられているシンプルな深層状態空間モデルでは複雑な環境を扱う問題設定に上手くスケールしない問題を示した.その上で深層状態空間モデルの状態表現の階層性を明示的にモデル化した提案手法によってより高次元の状態表現を扱えるようにし,さらに映像予測の性能が向上することを示した.
実験では,定性的な大きな優位性は示せなかったものの,高次元の状態変数を用いた学習を可能にしたことは深層状態空間モデルの大きな問題を克服したと言える.これにより,これまで映像予測の分野では実機ロボットへの応用上の制約が多いにも関わらず自己回帰的なモデルの研究が主流であったが,状態空間モデルベースの手法が見直されるきっかけになるかもしれない.

第\ref{chap:discussion}章では展望として深層状態空間モデルの研究の方向性を複数上げたが,これらの研究をすすめることによってより高性能な予測が可能になり,また実機への応用の可能性も高められると考える.さらに応用の例として,映像予測の実機応用に加え,新たな物理シミュレーションの近似アプローチと新たなロボット学習のあり方の可能性について述べた.この二つの応用例は現段階では可能性の話に過ぎず実現可能かは定かでないがどちらも実現すれば社会的な価値は大きいと考えられるので,今後も慎重に研究を継続していきたい.
