\chapter{考察}
\label{chap:discussion}

\section{本研究の貢献}

\section{今後の課題}
\subsection{転移学習}
環境のダイナミクスとして、普遍的な物理法則を一部獲得していると考えられることから、様々なタスクで
特に階層的なモデルでは、低次元のstateに大域的な物理法則に関するダイナミクス、高次元のstateに比較的表面的なダイナミクスを
獲得していると考えられ、普遍的な環境の多少の変化には

\section{社会応用}
\subsection{実機ロボットへの応用}

\subsection{物理シミュレーションの近似}
本研究で扱った深層学習ベースの行動条件付き映像予測では物理現象の結果として観測されるデータの近似を行うが、これは物理現象の近似という意味で現行の物理シミュレーションソフトウェアと同じことを行っていると解釈ができる。現在、物理シミュレーションの手法としては、既知の様々なスケールの物理法則を記述し微小時間・微小空間単位で逐次的に各領域の状態を計算し全体の結果を予測することが一般的である。しかしこのアプローチは、物理法則が既知である必要があり、また複雑な物理法則に対しては予測に膨大な時間がかかりリアルタイムに予測ができないという問題がある。例えば[]は、蕎麦にオイスターソースをかけて混ぜる物理シミュレーションを扱っているが、1フレームごとの見た目はとてもリアルなものの物理現象としては以前不自然な部分があり、さらに30fpsで1秒の予測をするのに29時間かかると報告している.現行の物理演算ベースの映像予測に対し深層学習ベースの映像予測は、物理法則の正しさや多視点から見た際の一貫性が保証できないなど機能として制限は多いものの、一度学習すればリアルタイムで予測を行うことができる。また必要なデータを集めることで機能の制限を解消することもできるはずである.

これからの人工知能の研究には仮想現実環境の開発が欠かせない。これは、昨今強化学習やロボット学習の分野で相次いて世界中の研究機関が学習用のシミュレーション環境[Meta world, control suit, RLBench]を開発し公開していることからも伺えるが、物理的なタスクを解けるよう学習するには実環境では危険であったり学習の並列化が難しいからである。自ら動いてものとの相互作用を繰り返す中で身体性また予期しなかった、身体性という観点から.

てと考えているが、物理演算ベースと深層学習ベースの物理シミュレーションの融合、あるいは深層学習ベースによる代替を研究していくことも重要になるだろう.

