\chapter{序論}
\label{chap:introduction}
\section{本研究の背景}

\subsection{ロボット学習}

多様な環境でさまざまなタスクが遂行可能な汎用的なロボット(generalist robots)の開発は,ロボット工学の最重要課題の一つである.
ロボットハードウェアの低価格化、汎用的なロボットソフトウェアの普及に加え、近年の急速な深層学習技術の発展を受けて、ロボットの制御方策を自ら学習させる{\bf ロボット学習}の研究が進んでおり,少しずつ遂行可能なタスクが増えている。
ロボット学習は問題設定によって様々な手法があるが、例えば強化学習をベースにして複雑な形状の物体の把持方策を(grasp2vec)学習するものや、模倣学習によって人のデモンストレーションを真似て食器の出し入れを学習するもの(TCN)、更にはDNNベースを含め様々な手法を統合して家庭用ロボットによる片付け \cite{hatori2018interactively}を遂行する手法などが提案されている。このように実用的なタスクも解決できつつあり、自動化が進む社会においてその活躍の期待値が高まっている。

\subsection{状態表現学習}
(https://arxiv.org/pdf/1802.04181.pdf)データを特徴づける情報を抽出する
DNNは特徴抽出が得意.教師なし学習で表現学習と呼ばれる.時間変化する環境に拡張した表現学習のことで,各時刻の生の観測データやその系列からこの状態表現(特徴)を見つけることが目的となる.

planet dreamer
上記のような強化学習や実機タスクへの応用から(行動条件付き)映像予測だけを切り取った研究も多い.

\section{本研究の目的}


提案手法の有効性について定性的・定量的な評価を行う.

最後に実験結果を踏まえて,今後の課題と社会応用について述べる.

\section{本論文の構成}
本論文の構成は以下のとおりである.


では
% 第\ref{chap:prerequisite}章では,本研究の前提となる深層生成モデルやメタ学習についての知識を述べる.

% 第\ref{chap:meta_gqn}章では,生成クエリネットワークの確率モデルについてメタ学習のフレームワークを用いて考察を行い,問題点を指摘する.

% %第\ref{chap:related}章では,を行う.
% 第\ref{chap:proposal}章では,前章の議論を踏まえて,生成クエリネットワークを改善する提案手法について説明する.

% 第\ref{chap:experiment}章では,提案手法に対する評価実験の結果を示し,実験結果への考察を行う.

% 第\ref{chap:discussion}章では,本研究の今後の課題と具体的な社会応用可能性について議論する.

% 第\ref{chap:conclusion}章を本論文のまとめとする.