\chapter{実験}
\label{chap:experiment}
本研究では,第\ref{chap:proposal}章で述べた提案手法の有効性を検証するために,を用いて,評価実験を行った.本章では,行なった実験の内容について説明した後,実験結果について述べていく.最後に実験結果を踏まえた考察を行う.

\section{実験内容}
潜在変数の次元は、5章で示すように学習がある程度進むことが確認できた〇〇64次元~〇〇次元をベースラインとして設定する.

\subsection{実験概要}
adam
評価指標には,予測誤差(負の対数尤度),hogehogeスコアを用いる
push datasetを使う
本実験の

\subsection{BAIR Push Dataset}
\subsection{hogehogeスコア}
オリジナル/ノンオリジナル
\subsection{モデルアーキテクチャ}
\subsection{}
\subsection{実装}
pytorch + Pixyz, tensorflow dataset
学習時のテクニック
\begin{itemize}
    \item リパらトリック
    \item min stddev
    \item share
\end{itemize}

\section{実験結果}
\subsection{定量評価(尤度)}

安定した.

潜在変数のサンプリングが安定し,KL項が安定して下がりやすくなったからだと思ふ

尤度が下がった

\subsection{定量評価(hogehogeスコア)}
\subsection{定性評価}
きれいになってくれ