\chapter*{概要}
\addcontentsline{toc}{chapter}{概要}
\label{chap:abstract}

近年,深層ニューラルネットワーク(DNN)を使用した機械学習手法の発展を背景としてロボットの制御方策を自ら学習させるロボット学習の研究が進んでいる.ロボット学習において未来の観測,特に自らの行動に対するフィードバックを適切に予測することは,より良い制御方策の獲得に繋がえる上,安全性の評価などに欠かせない.時間変化する環境のDNNを用いたモデル化手法として深層状態空間モデル(DSSM: Deep State Space Model)がありこれは昨今深層強化学習の分野で取り入れられているが,DSSMを用いて映像予測を学習することで良い環境の遷移モデルを獲得できることがわかっている.しかし現在一般的に使われているDSSMとその学習方法は各時刻の状態が低次元のベクトルで表現できるという仮定を暗に置いており,高次元な表現を仮定すると学習が進まず複雑な環境の予測問題にスケールさせる方法は自明でない.本研究ではまずこのような問題があることを実験的に指摘した上で,状態の表現に階層性を考えることでDSSMにより高次元な状態ベクトルを使っても学習が進むよう改善し,さらに低次元な状態ベクトルを用いた場合よりも良い表現が獲得できることを示す.評価実験では様々な物体が置かれた机上でのロボットマニピュレーションを題材にしたデータセットを用いてロボットの行動で条件付けた映像予測を行う.本研究では単純なDSSMをベースラインに設定し,提案手法の有効性を示した上で,研究の展望と社会応用の可能性について述べる.



% 近年,深層ニューラルネットワーク(DNN)を使用した機械学習手法の発展を背景としてロボットの制御方策を自ら学習させるロボット学習の研究が進んでいる.
% ロボット学習において未来の観測,特に自らの行動に対するフィードバックを適切に予測するよう学習することは,環境の遷移モデル(内部モデル)を獲得すること,そしてそれを用いたプランニングや行動結果の予測,安全性の評価などに欠かせない.
% 時間変化する環境のDNNを用いたモデルとして深層状態空間モデル(Deep State Space Model)があり,これは昨今深層強化学習の分野で取り入れられているが,深層状態空間モデルを用いて映像予測を学習することで内部モデルが獲得できるとされている\cite{Gregor2015}.
% しかし現在一般的に使われて深層状態空間モデルとその学習方法は各時刻の状態が低次元のベクトルで表現できるという仮定を暗に置いており,高次元な表現を仮定すると学習が進まず,複雑な環境の予測問題にスケールさせる方法は自明でない.

% 本研究ではまずこのような問題があることを実験的に指摘した上で,状態の表現に階層性を考えることで深層状態空間モデルにより高次元な状態表現を仮定できるように拡張し,低次元な表現を仮定した場合よりも良い表現が獲得できることを示す.評価実験では様々な物体が置かれた机上でのマニピュレーションを題材にしたデータセットを用いてロボットの行動で条件付けた映像予測を行う.単純な深層状態空間モデルをベースラインに設定し,提案手法の定性的・定量的な有効性を示す.